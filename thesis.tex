%% These files were modified in Lyx by:
%           Lus F. C. Fgueiredo, Fev., 2017.
%% Git repo for Lyx version are available at:
%%          https://github.com/lara-unb/DOCUMENTS/tree/master/templates
%% Latex version for sharelatex and overleaf v2 ported by:
%%          Vinícius Galvão Guimarães
%% 
%% Contributions of:
%%          Miguel Eduardo Gutierrez Paredes
%% LyX 2.2.3 created this file.  For more info, see http://www.lyx.org/.

%\documentclass[11pt,oneside,portuguese]{book}
\documentclass[11pt,oneside,english]{book}
\usepackage[utf8]{inputenc}
\setcounter{secnumdepth}{3}
\setcounter{tocdepth}{3}
\usepackage{array}
\usepackage{verbatim}
\usepackage{amsmath}
\usepackage{amsthm}
\usepackage{xargs}[2008/03/08]
\usepackage{import}
\usepackage{siunitx}
%\usepackage[justification=centering]{caption} % Center all captions...

% Used for Code  blocks, comment if not used...
\usepackage{listings}
\lstdefinestyle{global}{%
	basicstyle=\ttfamily\scriptsize\color{black!90},%
	stringstyle=\itshape\color{magenta},%
	showstringspaces=false,%
	keywordstyle={[1]\bfseries\color{keycolor}},%
	keywordstyle={[2]\bfseries\color{mbleu}},%
	commentstyle=\color{blue}\slshape,%
	framexleftmargin=1mm,%
	backgroundcolor=\color{black!2},%
	numbers=left,%
	stepnumber=1,%
	numberstyle=\color{Gray},
}

\lstdefinestyle{makefile}{%
	otherkeywords={.SUFFIXES},
	alsoletter={:},
	morekeywords=[1]{SUFFIX, CPP_},
	morekeywords=[2]{vasp:,makeparam:,zgemmtest:,dgemmtest:,ffttest:,kpoints:,clean:},
	style=global,%
	morecomment=[l][commentstyle]{\#},%
	emphstyle={\color{vimvert}},%
	moredelim=[s][\color{vimvert}]{\$(}{)}%
}

\lstdefinestyle{C}{%
	belowcaptionskip=1\baselineskip,
	breaklines=true,
	style=global,%
	language=C,
	showstringspaces=false,
	basicstyle=\footnotesize\ttfamily,
	keywordstyle=\bfseries\color{green!40!black},
	commentstyle=\itshape\color{purple!40!black},
	identifierstyle=\color{blue},
	stringstyle=\color{orange},
}

% acronyms
\usepackage[acronym]{glossaries}


\makeatletter
\makeglossaries

%%%%%%%%%%%%%%%%%%%%%%%%%%%%%% LyX specific LaTeX commands.
\providecommand{\LyX}{L\kern-.1667em\lower.25em\hbox{Y}\kern-.125emX\@}
%% Because html converters don't know tabularnewline
\providecommand{\tabularnewline}{\\}

%%%%%%%%%%%%%%%%%%%%%%%%%%%%%% Textclass specific LaTeX commands.
    
 	\usepackage{ft4unb}
  \theoremstyle{plain}
  \ifx\thechapter\undefined
    \newtheorem{thm}{\protect\theoremname}
  \else
    \newtheorem{thm}{\protect\theoremname}[chapter]
  \fi
  \theoremstyle{definition}
  \ifx\thechapter\undefined
    \newtheorem{example}{\protect\examplename}
  \else
    \newtheorem{example}{\protect\examplename}[chapter]
  \fi
  \theoremstyle{definition}
  \ifx\thechapter\undefined
    \newtheorem{defn}{\protect\definitionname}
  \else
    \newtheorem{defn}{\protect\definitionname}[chapter]
  \fi
  \theoremstyle{plain}
  \ifx\thechapter\undefined
    \newtheorem{lem}{\protect\lemmaname}
  \else
    \newtheorem{lem}{\protect\lemmaname}[chapter]
  \fi

%%%%%%%%%%%%%%%%%%%%%%%%%%%%%% User specified LaTeX commands.
%%% PACOTES: FONTE  (VERIFICAR SE ESTÃO INSTALADOS)
\usepackage[T1]{fontenc}


%%% PACOTES: GRAFICOS, TABELAS ETC
\usepackage{graphicx}
  % declare the path(s) where your graphic files are
  \graphicspath{{./figs/}}
  % and their extensions so you won't have to specify these with
  % every instance of \includegraphics
  \DeclareGraphicsExtensions{.pdf,.jpeg,.png,.jpg,.bmp}
\usepackage{multicol}
\usepackage{subfig}
\usepackage{array}

%%% PACOTES: ALGORITMOS
\usepackage{algorithm}
\usepackage{algpseudocode}

%%% PACOTES: OUTROS
\usepackage{pdfpages}


%%% PACOTES NECESSÁRIOS PELO TEMPLATE (JÁ INSTALADOS):
%%% color, [usenames,dvipsnames,svgnames,table]{xcolor}
%%% {eso-pic,graphicx}, mdframed, tcolorbox
%%% times, geometry, import, ifthen, calc,{xstring,xifthen}

\input{setup/other}
\input{setup/theorem_lemma}
\usepackage{babel}
  \providecommand{\definitionname}{Definition}
  \providecommand{\examplename}{Example}
  \providecommand{\lemmaname}{Lemma}
\providecommand{\theoremname}{Theorem}
  \addto{\captionsenglish}{%
    \renewcommand{\bibname}{References}
    \renewcommand{\contentsname}{SUMMARY}
    \renewcommand{\listfigurename}{LIST OF FIGURES}
    \renewcommand{\listtablename}{LIST OF TABLES}
    \renewcommand{\bibname}{BIBLIOGRAPHY}
  }
  \addto\captionsportuguese{%
    \renewcommand{\bibname}{REFER\^{E}NCIAS BIBLIOGR\'{A}FICAS}
    \renewcommand{\contentsname}{SUM\'{A}RIO}
    \renewcommand{\listfigurename}{LISTA DE FIGURAS}
    \renewcommand{\listtablename}{LISTA DE TABELAS}
  }


% Do not change the sequence of this package import.
% hyperref must be on of the last to be imported and abntex must be after it.
% DON`t CHANGE THIS
\usepackage[hidelinks]{hyperref}
\usepackage[num]{abntex2cite}
% relative to abntex2. cite mode
\citebrackets[]
\input{setup/listing_def}

% To use this acro list, you have to us them in the text as they are listed in the acro_list.tex.
% Example for Glossary:
% \newglossaryentry{latex}
% {
%     name=latex,
%     description={Is a mark up language specially suited 
%     for scientific documents}
% }
 
\newglossaryentry{maths}
{
    name=mathematic,
    description={Mathematics is what mathematicians do}
}

% Example for Acronyms:
% \newacronym{label}{Acronym}{Full name}
\newacronym{uti}{UTI}{Unidade de Tratamento Intensivo}
\newacronym{scuba}{SCUBA}{Self-Contained Underwater Breathing Apparatus}
\newacronym{radar}{RADAR}{RAdio Detection And Ranging}

% In the text:
%     use \gls{label}, for an first time extend name, plus acronyms
%     use \acrshort{label}, for an acronym only
%     use \acrlong{label}, for a full name only
% In the text , for plural:
%     \glspl{duck}
% In the text , for Caps and/or plural:
%     \Gls{duck}
%     \Glspl{duck}




\makeatother
\begin{document}

\titulolinhas{}

%% [sub-título]{Título original}
\titulolinhas[Overleaf V2]{Template em LaTeX para dissertações da UnB/FT:}

\maketitle

%% No caso de haver outros autores (máximo 3), crie mais um "\autores"
\autores{Margaret Hamilton}
\autores{Minch Yoda}

%% {Grau desejado em detalhes}{Tipo de monografia}{Departamento}{Grau desejado}{Sigla do departamento}{Programa do aluno ou departamento novamente]}{Nome da faculdade}{Sigla da Faculdade}
\grau{Doutor}{Tese de doutorado}{Engenharia El\'{e}trica}{ENE}{Engenharia de Sistemas Eletrônicos e de Automação}{Faculdade de Tecnologia}{FT}{Electronic and Automation Systems Engineering}

\datainfo{Junho}{28}{2019}
\datainfoAnotherLang{June}{28th}

%% {Traduzido para outro língua}{Título original}
\titulolinguas{LaTeX Template for thesis dissertation of UnB/FT: Overleaf v2 Version}{Template em Latex para dissertações do UnB/FT: Overleaf V2}

%% Senão houver co-orientador, deixe o campo [] vazio.
\orientador[Who, Dr.]{Strange, Dr.}

%% Para adicionar mais membros da banca: Após o último membro, crie uma nova linha (enter) e vá na aba superior esquerda (abaixo de “file”) e selecione “membro da banca”. Para deletar membros, apenas delete o item correspondente
\membrodabanca[Orientador]{Prof. Dr. Strange \textendash{} ENE/Universidade de Brasília}

\membrodabanca[Membro Interno]{Prof. Dr. Jhon H. Watson \textendash{} Dep./Universidade}

\membrodabanca[Membro Externo]{Dr. Evil \textendash{} Dep./Universidade}

\catalogonome[Hamilton, M., Yoda, M.]{Hamilton, Margaret; Yoda, Minch}

%% {Publicação N#}{Palavra chave #1}{chave #2}{chave #3}{chave #4}
\catalogoinfo{PPGEA.TD-001/11}{Palavra 1}{Palavra 2}{Palavra 3}{Palavra 4}{Keyword 1}{Keyword 2}{Keyword 3}{Keyword 4}
\input{parts/dedicated_to}

\runfrontend{}

\import{}{macros}
\import{parts/}{abstract}

\sumario

\listadefiguras

\listadetabelas

\listadeacronimos

% Também é possível expandir o formato dos acrônimos para uma lista de símbolos. Para isso, procure como usar o pacote glossary e ative o comando abaixo.
%\listadesimbolos

\markboth{}{}

% Caso nãos use o glossary para símbolos, voĉe pode utilizar o arquivos "symbols.tex" da forma mais simples.
\import{parts/}{symbols}

% A mesma troca do glossary, pode ser feita para notation. Atualize o nome no arquivo "languages.tex".
\import{parts/}{notation}

%CORPO PRINCIPAL
\mainmatter 
\setcounter{page}{1} \pagestyle{plain} 

\import{parts/}{capt_01}
\import{parts/}{capt_02}

%BIBLIOGRAPHY
\addcontentsline{toc}{chapter}{REFERENCES} 

%\bibliographystyle{chicago}
%\bibliographystyle{IEEEtran}
%\bibliographystyle{acm}
\bibliography{bibliography}

\part*{APPENDIX}
\appendix
\import{parts/}{appendix1}

\end{document}
